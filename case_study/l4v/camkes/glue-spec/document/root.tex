%
% Copyright 2014, NICTA
%
% This software may be distributed and modified according to the terms of
% the GNU General Public License version 2. Note that NO WARRANTY is provided.
% See "LICENSE_GPLv2.txt" for details.
%
% @TAG(NICTA_GPL)
%

\documentclass[11pt,a4paper]{scrreprt}

% These old font commands have been removed from newer versions of
% the scrreprt document class, but isabelle.sty still uses them.
\DeclareOldFontCommand{\rm}{\normalfont\rmfamily}{\mathrm}
\DeclareOldFontCommand{\sf}{\normalfont\sffamily}{\mathsf}
\DeclareOldFontCommand{\tt}{\normalfont\ttfamily}{\mathtt}
\DeclareOldFontCommand{\bf}{\normalfont\bfseries}{\mathbf}
\DeclareOldFontCommand{\it}{\normalfont\itshape}{\mathit}
\DeclareOldFontCommand{\sl}{\normalfont\slshape}{\@nomath\sl}
\DeclareOldFontCommand{\sc}{\normalfont\scshape}{\@nomath\sc}

\newif \ifDraft         \Draftfalse

\usepackage{isabelle,isabellesym}

% further packages required for unusual symbols (see also
% isabellesym.sty), use only when needed

%\usepackage{amssymb}
  %for \<leadsto>, \<box>, \<diamond>, \<sqsupset>, \<mho>, \<Join>,
  %\<lhd>, \<lesssim>, \<greatersim>, \<lessapprox>, \<greaterapprox>,
  %\<triangleq>, \<yen>, \<lozenge>

\usepackage[english]{babel}
  %option greek for \<euro>
  %option english (default language) for \<guillemotleft>, \<guillemotright>

%\usepackage[only,bigsqcap]{stmaryrd}
  %for \<Sqinter>

%\usepackage{eufrak}
  %for \<AA> ... \<ZZ>, \<aa> ... \<zz> (also included in amssymb)

%\usepackage{textcomp}
  %for \<onequarter>, \<onehalf>, \<threequarters>, \<degree>, \<cent>,
  %\<currency>


% Extra CAmkES bits.
\usepackage{graphicx}
\usepackage{enumerate}

% From ERTOS setup
\setkeys{Gin}{keepaspectratio=true,clip=true,draft=false,width=\linewidth}
\usepackage{times,cite,url,fancyhdr,microtype,color,geometry}

\renewcommand{\ttdefault}{cmtt}	% CM rather than courier for \tt

\usepackage{xspace}
\usepackage{listings}
\newcommand{\camkeslisting}[1]{{\lstset{basicstyle=\small\ttfamily,keywordstyle=\bf,morekeywords={assembly,component,composition,connection,control,consumes,dataport,emits,event,from,in,inout,int,out,procedure,provides,smallstring,to,uses}}\lstinputlisting{#1}}}


\ifDraft
\usepackage{draftcopy}
\newcommand{\Comment}[1]{\textbf{\textsl{#1}}}
\newcommand{\FIXME}[1]{\textbf{\textsl{FIXME: #1}}}
\newcommand{\todo}[1]{\textbf{TODO: \textsl{#1}}}
\date{\small\today}
\else
\newcommand{\Comment}[1]{\relax}
\newcommand{\FIXME}[1]{\relax}
\newcommand{\todo}[1]{\relax}
\date{}
\fi


% From camkes manual
\newcommand{\selfour}{seL4\xspace}
\newcommand{\Selfour}{SeL4\xspace}
\newcommand{\camkes}{CAmkES\xspace}

\newcommand{\code}[1]{\texttt{#1}}


\newcommand{\titl}{CAmkES Glue Code Semantics}
\newcommand{\authors}{Matthew Fernandez, Peter Gammie, June Andronick, Gerwin Klein, Ihor Kuz}

\definecolor{lcol}{rgb}{0,0,0.5}
\usepackage[bookmarks,hyperindex,pdftex,
            colorlinks=true,linkcolor=lcol,citecolor=lcol,
            filecolor=lcol,urlcolor=lcol,
            pdfauthor={\authors},
            pdftitle={\titl},
            plainpages=false]{hyperref}


\addto\extrasenglish{%
\renewcommand{\chapterautorefname}{Chapter}
\renewcommand{\sectionautorefname}{Section}
\renewcommand{\subsectionautorefname}{Section}
\renewcommand{\subsubsectionautorefname}{Section}
\renewcommand{\appendixautorefname}{Appendix}
\renewcommand{\Hfootnoteautorefname}{Footnote}
}

% urls in roman style
\urlstyle{rm}

\lstset{basicstyle=\small\tt}

% isabelle style
\isabellestyle{tt}

% for uniform isabelle font size
\renewcommand{\isastyle}{\isastyleminor}

% Abstract various things that might change.
\newcommand{\ccode}[1]{\texttt{#1}}
\newcommand{\isabelletype}[1]{\emph{#1}}
\newcommand{\isabelleterm}[1]{\emph{#1}}

\renewcommand{\isamarkupheader}[1]{\chapter{#1}}
\renewcommand{\isamarkupchapter}[1]{\chapter{#1}}
\renewcommand{\isamarkupsection}[1]{\section{#1}}
\renewcommand{\isamarkupsubsection}[1]{\section{#1}}
\renewcommand{\isamarkupsubsubsection}[1]{\subsection{#1}}

\newcommand{\nictafundingacknowledgement}{%
% Lifted from http://wiki.inside.nicta.com.au/display/CHNOPSR/Funding+Acknowledgement 28-10-2013
NICTA is funded by the Australian Government through the Department of Communications and the Australian Research Council through the ICT Centre of Excellence Program. NICTA is also funded and supported by the Australian Capital Territory, the New South Wales, Queensland and Victorian Governments, the Australian National University, the University of New South Wales, the University of Melbourne, the University of Queensland, the University of Sydney, Griffith University, Queensland University of Technology, Monash University and other university partners.}

\newcommand{\ABN}{ABN 62 102 206 173}

\newcommand{\cpright}{Copyright \copyright\ 2013 NICTA, \ABN.  All rights reserved.}
\newcommand{\disclaimer}{%
\cpright}

\newcommand{\trdisclaimer}{%
This material is based on research sponsored by Air Force Research Laboratory
and the Defense Advanced Research Projects Agency (DARPA) under agreement number
FA8750-12-9-0179. The U.S. Government is authorized to reproduce and distribute
reprints for Governmental purposes notwithstanding any copyright notation
thereon.

The views and conclusions contained herein are those of the authors and should
not be interpreted as necessarily representing the official policies or
endorsements, either expressed or implied, of Air Force Research Laboratory,
the Defense Advanced Research Projects Agency or the U.S.Government.}

\newcommand{\smalldisclaimer}{}
\newcommand{\bigdisclaimer}{%
\nictafundingacknowledgement\\

\cpright\\

\vspace{2ex}
\noindent\trdisclaimer}

\newcommand{\pgstyle}{%
\fancyhf{}%
\renewcommand{\headrulewidth}{0pt}%
\fancyfoot[C]{}%
\fancyfoot[L]{\smalldisclaimer}%
\fancyfoot[R]{\sl\thepage}}

\fancypagestyle{plain}{\pgstyle}


\begin{document}

\parindent 0pt\parskip 0.5ex plus 0.2ex minus 0.1ex

%--------- title page
\newgeometry{left=25mm,right=25mm,top=35mm,bottom=35mm}

\begin{center}
 \includegraphics[width=0.2\textwidth]{imgs/NICTA_logo}
\end{center}
	
\vspace{14ex}
\textsf{\huge \titl}

%\vspace{2ex}
%\textsf{\huge \subtitl}

\vspace{4ex}
\rule{0.85\textwidth}{5pt}
\vspace{4ex}
  
{\large \authors 

\vspace{2ex}
April 2013}

\vfill
{\small
\bigdisclaimer
}

\thispagestyle{empty}
\newpage
~
\restoregeometry

\fancypagestyle{empty}{\pgstyle}
\pagestyle{empty}

%--------- end title page

\cleardoublepage

\chapter*{Abstract}

This document describes the formal dynamic semantics of \camkes glue code, in
particular of the communication stubs generated for components at compile
time. The semantics is based on a simple concurrent imperative language with
message passing that is easy to extend and instantiate for specific
applications. Instead of one generic semantics for all systems, we take the
approach of generating a high-level semantic description for each specific
ADL component specification to ease verification of specific systems in
the future.

We show the definitions and types for expressing components and glue code, and
provide some examples of generated Isabelle theories with synchronous,
asynchronous, and shared memory communication.


\cleardoublepage
\tableofcontents

%
% Copyright 2014, NICTA
%
% This software may be distributed and modified according to the terms of
% the GNU General Public License version 2. Note that NO WARRANTY is provided.
% See "LICENSE_GPLv2.txt" for details.
%
% @TAG(NICTA_GPL)
%

\chapter{Introduction}
CAmkES is a component platform for embedded microkernel-based systems, offering
many of the standard features available in component platforms.
Some relevant features of CAmkES that are not common to all component platforms
are:
\begin{itemize}
  \item \emph{Explicit composite components.} CAmkES components can be
    assembled to form a re-usable composite, that can then be referenced within
    a containing system.
  \item \emph{Multiple instantiation of a single component.} Multiple copies of
    a component can exist within a system, distinguished by different
    identifiers.
  \item \emph{Multiple implementations of an interface.} A single component can
    implement an interface more than once. This can be useful for providing a
    dedicated interface for each client or functionally different
    implementations.
  \item \emph{Distinction between active and passive components.} Components
    can have a thread of control or be invoked via an event loop. CAmkES
    distinguishes these modes of operation at an architectural level.
  \item \emph{``Provides'' interfaces can be left unsatisfied at runtime.} When
    a component implements a given interface, it does not have to be connected
    to a component that uses that interface. The converse is not true.
  \item \emph{Hardware devices are components.} Hardware devices can be
    specified in ADL that generates interface methods for accessing the device.
\end{itemize}


% generated text of all theories
\input{session}

% optional bibliography
\bibliographystyle{plain}
\bibliography{root}

\end{document}
