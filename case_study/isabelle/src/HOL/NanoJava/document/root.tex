\documentclass[11pt,a4paper]{article}
\usepackage{graphicx,latexsym,isabelle,isabellesym,latexsym,pdfsetup}

\urlstyle{tt}
\pagestyle{myheadings}

\addtolength{\hoffset}{-1,5cm}
\addtolength{\textwidth}{4cm}
\addtolength{\voffset}{-2cm}
\addtolength{\textheight}{4cm}

%remove spaces from the isabelle environment (trivlist makes them too large)
\renewenvironment{isabelle}
{\begin{isabellebody}}
{\end{isabellebody}}

\newcommand{\nJava}{\it NanoJava}

%remove clutter from the toc
\setcounter{secnumdepth}{3}
\setcounter{tocdepth}{2}

\begin{document}

\title{\nJava}
\author{David von Oheimb \\ Tobias Nipkow}
\maketitle

\begin{abstract}\noindent
  These theories define {\nJava}, a very small fragment of the programming 
  language Java (with essentially just classes) derived from the one given 
  in \cite{NipkowOP00}.
  For {\nJava}, an operational semantics is given as well as a Hoare logic,
  which is proved both sound and (relatively) complete. 
  The Hoare logic supports side-effecting expressions and
  implements a new approach for handling auxiliary variables.
  A more complex Hoare logic covering a much larger subset of Java is described
  in \cite{DvO-CPE01}.\\
See also the homepage of project Bali at \url{http://isabelle.in.tum.de/Bali/}
and the conference version of this document \cite{NanoJava}.
\end{abstract}

\tableofcontents
\parindent 0pt \parskip 0.5ex

\begin{center}
  \includegraphics[scale=0.7]{session_graph}  
\end{center}

\newpage
\input{session}

\newpage
\nocite{*}
\bibliographystyle{abbrv}
\bibliography{root}

\end{document}
