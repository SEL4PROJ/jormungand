
\documentclass[a4paper,fleqn]{article}

\usepackage{latexsym,graphicx}
\usepackage[refpage]{nomencl}
\usepackage{iman,extra,isar}
\usepackage{isabelle,isabellesym}
\usepackage{style}
\usepackage{mathpartir}
\usepackage{amsthm}
\usepackage{pdfsetup}

\newcommand{\cmd}[1]{\isacommand{#1}}

\newcommand{\isasymINFIX}{\cmd{infix}}
\newcommand{\isasymLOCALE}{\cmd{locale}}
\newcommand{\isasymINCLUDES}{\cmd{includes}}
\newcommand{\isasymDATATYPE}{\cmd{datatype}}
\newcommand{\isasymDEFINES}{\cmd{defines}}
\newcommand{\isasymNOTES}{\cmd{notes}}
\newcommand{\isasymCLASS}{\cmd{class}}
\newcommand{\isasymINSTANCE}{\cmd{instance}}
\newcommand{\isasymLEMMA}{\cmd{lemma}}
\newcommand{\isasymPROOF}{\cmd{proof}}
\newcommand{\isasymQED}{\cmd{qed}}
\newcommand{\isasymFIX}{\cmd{fix}}
\newcommand{\isasymASSUME}{\cmd{assume}}
\newcommand{\isasymSHOW}{\cmd{show}}
\newcommand{\isasymNOTE}{\cmd{note}}
\newcommand{\isasymCODEGEN}{\cmd{code\_gen}}
\newcommand{\isasymPRINTCODETHMS}{\cmd{print\_codethms}}
\newcommand{\isasymFUN}{\cmd{fun}}
\newcommand{\isasymFUNCTION}{\cmd{function}}
\newcommand{\isasymPRIMREC}{\cmd{primrec}}
\newcommand{\isasymRECDEF}{\cmd{recdef}}

\newcommand{\qt}[1]{``#1''}
\newcommand{\qtt}[1]{"{}{#1}"{}}
\newcommand{\qn}[1]{\emph{#1}}
\newcommand{\strong}[1]{{\bfseries #1}}
\newcommand{\fixme}[1][!]{\strong{FIXME: #1}}

\newtheorem{exercise}{Exercise}{\bf}{\itshape}
%\newtheorem*{thmstar}{Theorem}{\bf}{\itshape}

\hyphenation{Isabelle}
\hyphenation{Isar}

\isadroptag{theory}
\title{Defining Recursive Functions in Isabelle/HOL}
\author{Alexander Krauss}

\isabellestyle{tt}

\begin{document}

\date{\ \\}
\maketitle

\begin{abstract}
  This tutorial describes the use of the \emph{function} package,
	which provides general recursive function definitions for Isabelle/HOL.
	We start with very simple examples and then gradually move on to more
	advanced topics such as manual termination proofs, nested recursion,
	partiality, tail recursion and congruence rules.
\end{abstract}

%\thispagestyle{empty}\clearpage

%\pagenumbering{roman}
%\clearfirst

%
% Copyright 2014, NICTA
%
% This software may be distributed and modified according to the terms of
% the GNU General Public License version 2. Note that NO WARRANTY is provided.
% See "LICENSE_GPLv2.txt" for details.
%
% @TAG(NICTA_GPL)
%

\chapter{Introduction}
CAmkES is a component platform for embedded microkernel-based systems, offering
many of the standard features available in component platforms.
Some relevant features of CAmkES that are not common to all component platforms
are:
\begin{itemize}
  \item \emph{Explicit composite components.} CAmkES components can be
    assembled to form a re-usable composite, that can then be referenced within
    a containing system.
  \item \emph{Multiple instantiation of a single component.} Multiple copies of
    a component can exist within a system, distinguished by different
    identifiers.
  \item \emph{Multiple implementations of an interface.} A single component can
    implement an interface more than once. This can be useful for providing a
    dedicated interface for each client or functionally different
    implementations.
  \item \emph{Distinction between active and passive components.} Components
    can have a thread of control or be invoked via an event loop. CAmkES
    distinguishes these modes of operation at an architectural level.
  \item \emph{``Provides'' interfaces can be left unsatisfied at runtime.} When
    a component implements a given interface, it does not have to be connected
    to a component that uses that interface. The converse is not true.
  \item \emph{Hardware devices are components.} Hardware devices can be
    specified in ADL that generates interface methods for accessing the device.
\end{itemize}

\input{Functions.tex}
%\section{Conclusion}

\fixme{}






\begingroup
%\tocentry{\bibname}
\bibliographystyle{plain} \small\raggedright\frenchspacing
\bibliography{manual}
\endgroup

\end{document}


%%% Local Variables: 
%%% mode: latex
%%% TeX-master: t
%%% End: 
